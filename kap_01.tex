\chapter{Einleitung}
Wenn Softwaresysteme über Jahre hinweg unkontrolliert wachsen, werden Sie groß, unübersichtlich und schlecht zu managen. Vor allem, wenn eine Junge Firma mehr Wachstum erfährt als erwartet, wird häufig weniger Zeit in die gute, geordnete Planung von Software gesteckt, als in die reine Entwicklung von Features. Userwachstum führt zu erhöhter Last auf Systemen, die Skalierung fordert. User wollen neue Features und neue Features bedeuten neuen Code. Schnell kann es passieren, dass die Codebase mehr einem Flickenteppich ähnelt, als einem stabilen Softwaresystem. Änderungen am Quellcode und Deploy können eine Qual werden. Und selbst wenn Zeit und Arbeit in gute Architektur gesteckt wird, so gehen verschiedene Frameworks doch immer unterschiedlich damit um, Interdependenzen im Code führen zu vermischten Strukturen und bla bla bla.

Die große Herausforderung besteht dann darin, den Code managable zu machen um Entwickler- und Betreiberproduktivität wieder zu erhöhen. Oder, idealerweise, besteht das Ziel darin, ein System gar nicht erst in diesen Zustand kommen zu lassen. In beiden Fällen kann es hilfreich sein, den eigenen Architekturstil zu reflektieren.

Eine Architekturweise, die in den letzten Jahren immer mehr an Beliebtheit gewonnen hat, ist die ``Microservice Architecture''. Microservice Architecture verspricht, Code in kleine, gut zu handhabende Systemen aufzusplitten und somit klare Strukturen zu fördern. 

In der vorliegenden Arbeit werde ich den Architekturstil vorstellen, dessen Geschichte erläutern und die Vor- und Nachteile analysieren. Anhand eines praktischen Beispiels werde ich einen bestehende monolithische Anwendung um einen Microservice erweitern und meine Vorgehensweise erläutern. Abschließend werde ich Schlüsse zur Umsetzung eines Anwendungsteils mit Microservice Architecure ziehen und das Pattern bewerten.