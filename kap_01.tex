\chapter{Einleitung}
Wenn Softwaresysteme über Jahre hinweg unkontrolliert wachsen, werden sie gegebenenfalls unübersichtlich und schlecht zu managen. Vor allem, wenn eine Junge Firma mehr Wachstum erfährt als erwartet, wird häufig wenig Zeit in gute, geordnete Planung von Software gesteckt. Die reine Entwicklung von Features hat dann oft Vorrang. Schnelles Userwachstum führt zu erhöhter Last auf Systemen, Skalierung ist erforderlich. Benutzer fordern neue Features und neue Features bedeuten neuen Code. Schnell kann es passieren, dass die Codebase mehr einem Flickenteppich ähnelt, als einem stabilen Softwaresystem. Für neue Mitarbeiter wird es dann immer schwieriger sich in den bestehenden Code einzuarbeite. Interdependzen sind kaum mehr absehrbar. Gibt es vielleicht einen Hook der nach Änderungen an einem Model greift und etwas komplett anderes intiiert? Auch Deploys gestalten sich in großen Systemen schwieriger. Eine kleine Änderung an einer Stelle im Code erfordert einen redeploy der gesamten Anwendung. Je nach Grad der Automatisierung von Deploys kann dies zur Qual werden. Und selbst wenn Zeit und Arbeit in gute Architektur gesteckt wird, so gehen verschiedene Frameworks doch immer unterschiedlich mit Wachstum um, Interdependenzen im Code führen zu vermischten Strukturen und immer mehr Klassen die zwischen verschiedenen Teilanwendungen geteilt werden. Und oftmals wurde bei einer jungen Firma nunmal nicht die beste Technologie für die Zukunft gewählt. Der Fokus liegt am Anfang schließlich im Jetzt. Dann heißt es oftmals ``Refactor or Rewrite?''~\footcite[vgl.][]{refactorrewrite}.

Die große Herausforderung besteht darin, entweder die bestehende Codebaase wieder handhabbar zu machen, um Entwickler- und Betreiberproduktivität wieder zu erhöhen. Oder, idealerweise, ein System gar nicht erst in diesen Zustand kommen zu lassen. In beiden Fällen kann es hilfreich sein, den eigenen Architekturstil zu reflektieren.

Eine Architekturweise, die in den letzten Jahren immer mehr an Beliebtheit gewonnen hat, ist die ``Microservice Architektur''. Microservice Architektur verspricht, Code in kleine, gut zu handhabende Systemen aufzusplitten und somit klare Strukturen zu fördern. 

In der vorliegenden Arbeit werde ich den Stil der Microservice Architektur vorstellen, dessen Geschichte erläutern und die Vor- und Nachteile benennen. Anhand eines praktischen Beispiels werde ich einen bestehende monolithische Anwendung um einen Microservice erweitern und meine Vorgehensweise erläutern.
Abschließend werde ich Schlüsse zur Umsetzung eines Anwendungsteils mit Microservice Architecure ziehen und das Pattern bewerten.