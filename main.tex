%%%%%%%%%%%%%%%%%%%%%%%%%%%%%%%%%%%%%%%%%
%%            LMU-Vorlage              %%
%%                                     %%
%%         zur Erstellung einer        %%
%%   Dissertation mit pdflatex/latex   %%
%%                                     %%
%%  (2002) Robert Dahlke               %%
%%         & Sigmund Stintzing         %%
%%%%%%%%%%%%%%%%%%%%%%%%%%%%%%%%%%%%%%%%%

%\documentclass[12pt]{book}
\documentclass[a4paper,11pt,oneside]{book}

%%%%%%%%%%%%%%%%%%%%%%%%%%%%
%%   Zusaetzliche Pakete  %%
%%%%%%%%%%%%%%%%%%%%%%%%%%%%

\usepackage{a4wide}
\usepackage{fancyhdr}
\usepackage{graphicx}
\graphicspath{ {images/} }
\usepackage{wrapfig}
\usepackage{german}
\usepackage[bookmarks,german]{hyperref}
%\addto\extrasngerman{\def\figureautorefname{Abb.}}
%\renewcaptionname{german}\figureautorefname{Abb.}
\usepackage[utf8]{inputenc}
%\usepackage[style=authoryear]{biblatex}
\usepackage[backend=biber,style=authortitle-ibid]{biblatex}
\bibliography{bibliography}
\usepackage{listings}
% TODO: check Zeilenabstand und left-indent 4cm
\linespread{1.25}
\let\subsectionautorefname\sectionautorefname
\let\subsubsectionautorefname\sectionautorefname

%%%%%%%%%%%%%%%%%%%%%%%%%%%%%%
%% Definition der Kopfzeile %%
%%%%%%%%%%%%%%%%%%%%%%%%%%%%%%

\pagestyle{fancyplain}
\renewcommand{\chaptermark}[1]%
         {\markboth{\thechapter.\ #1}{}}
\renewcommand{\sectionmark}[1]%
         {\markright{\thesection\ #1}}
\lhead[\fancyplain{}{\bfseries\thepage}]%
    {\fancyplain{}{\bfseries\rightmark}}
\rhead[\fancyplain{}{\bfseries\leftmark}]%
\cfoot{}
\rfoot{{\bfseries\thepage}}


%%%%%%%%%%%%%%%%%%%%%%%%%%%%%%%%%%%%%%%%%%%%%%%%%%%%%
%%  Definition des Deckblattes und der Titelseite  %%
%%%%%%%%%%%%%%%%%%%%%%%%%%%%%%%%%%%%%%%%%%%%%%%%%%%%%

\newcommand{\LMUTitle}[9]{
  \thispagestyle{empty}
  \vspace*{\stretch{1}}
  {\parindent0cm
   \rule{\linewidth}{.7ex}}
  \begin{flushright}

    \vspace*{\stretch{1}}
    \sffamily\bfseries\Huge
    #1\\
    \vspace*{\stretch{1}}
    \sffamily\bfseries\large
    #2
    \vspace*{\stretch{1}}
  \end{flushright}
  \rule{\linewidth}{.7ex}
  \vspace*{\stretch{5}}
  \begin{center}
    \includegraphics[width=2in]{FHFLLogo}
  \end{center}
  \vspace*{\stretch{1}}
  \begin{center}\sffamily\LARGE{#5}\end{center}
  \newpage
  \thispagestyle{empty}

  \cleardoublepage
  \thispagestyle{empty}

  \vspace*{\stretch{1}}
  {\parindent0cm
  \rule{\linewidth}{.7ex}}
  \begin{flushright}
    \vspace*{\stretch{1}}
    \sffamily\bfseries\Huge
    #1\\
    \vspace*{\stretch{1}}
    \sffamily\bfseries\large
    #2
    \vspace*{\stretch{1}}
  \end{flushright}
  \rule{\linewidth}{.7ex}

  \vspace*{\stretch{2}}
  \begin{center}
    \Large Bachelorarbeit\\
    \Large an der Fachhochschule Flensburg\\
    \vspace*{\stretch{1}}
    \Large vorgelegt von\\
    \Large #2\\
    \vspace*{\stretch{1}}
    \large Erstgutachter:  #7 \\[1mm]
    \large Zweitgutachter: #8 \\[1mm]
    \vspace*{\stretch{1}}
    \Large Flensburg, den #6
  \end{center}

  \cleardoublepage
}




%%%%%%%%%%%%%%%%%%%%%%%%%%%%
%%  Beginn des Dokuments  %%
%%%%%%%%%%%%%%%%%%%%%%%%%%%%

\begin{document}

  \frontmatter


  \LMUTitle
      {Entwicklung eines Microservices zur Augmentierung einer bestehenden monolithischen Anwendung}               % Titel der Arbeit
      {Thomas Peikert}                       % Vor- und Nachname des Autors
      {Berlin}                             % Geburtsort des Autors
      {Angewandte Informatik}                         % Name der Fakultaet
      {Flensburg 2016}                          % Ort und Jahr der Erstellung
      {01.04.2016}                            % Tag der Abgabe
      {Prof. Dr. Milena Zachow}                          % Name des Erstgutachters
      {Prof. Dr. Hans-Werner Lang}                         % Name des Zweitgutachters
      {Pr"ufungsdatum}                         % Datum der muendlichen Pruefung


  \tableofcontents
  \markboth{Inhaltsverzeichnis}{Inhaltsverzeichnis}


  \listoffigures
  \markboth{Abbildungsverzeichnis}{Abbildungsverzeichnis}


  \listoftables
  \markboth{Tabellenverzeichnis}{Tabellenverzeichnis}
  \cleardoublepage


  \markboth{Zusammenfassung}{Zusammenfassung}
  \addcontentsline{toc}{chapter}{\protect Abstract}


\chapter*{Abstract}
Die vorliegende Bachelorarbeit beschäftigt sich mit der Entwicklung von Microservices als Alternative zur ``klassischen'', monolithischen Anwendungsentwicklung. Die integralen Bestandteile von ``Microservice Architektur'', sowie deren Vor- und Nachteile werden erläutert und anhand eines praktischen Beispiels umgesetzt.

Im Rahmen der Arbeit wird ein Microservice entwickelt, der eine bestehende monolithische Betriebsanwendung augmentieren soll. Im momentanen Setup werden Userprofildaten in eine Datenbank geschrieben und aus dieser ausgelesen. Aufgrund der hohen Komplexität dieser Daten und der großen Menge an Daten in der Datenbank, benötigen Profilqueries auf die Datenbank viel Zeit. Da das Auslesen dieser Daten ein integraler Bestandteil der Gesamtanwendung ist, soll hier eine Echtzeit-Schnittstelle bereitgestellt werden. Diese soll in Form einer separaten Anwendung in die bestehende Struktur eingebettet werden.

Hierfür soll ein leseoptimiertes Datenbankformat entwickelt werden, die Anwendung soll eine standardisierte Kommunikationsschnittstelle bieten und kann separat auf einer eigenen Infrastruktur deployt werden. Weiterhin muss eine Anbindung an die bestehende Anwendung geschaffen werden, sodass Änderungen an relevanten Daten in die Queryanwendung gelangen. Dies alles dient dem Zweck, Queries auf Userprofile in soweit zu beschleunigen, dass Anfragen auf Userprofildaten in Echtzeit geschehen können.


  \mainmatter\setcounter{page}{1}
  \chapter{Einleitung}
Wenn Softwaresysteme über Jahre hinweg wachsen, werden sie oft unübersichtlich und mit der Zeit häufig schlecht zu managen \cite{infaktuell}. Vor allem, wenn eine junge Firma mehr Wachstum erfährt als erwartet, wird oftmals zu wenig Zeit in gute, geordnete Planung von Software gesteckt. Die reine Entwicklung von Features hat dann oft Vorrang. Schnelles Userwachstum führt zu erhöhter Last auf Systemen, Skalierung ist erforderlich. Benutzer fordern neue Features und neue Features bedeuten neuen Code. Schnell kann es passieren, dass die Codebase mehr einem Flickenteppich ähnelt, als einem stabilen Softwaresystem. Für neue Mitarbeiter wird es dann immer schwieriger sich in den bestehenden Code einzuarbeiten. Interdependenzen sind kaum mehr absehbar. Auch Deploys gestalten sich in großen Systemen schwieriger. Eine kleine Änderung an einer Stelle im Code erfordert einen Redeploy der gesamten Anwendung. Je nach Grad der Automatisierung von Deploys kann dies zur Qual werden. Und selbst wenn Zeit und Arbeit in gute Architektur gesteckt wird, so gehen verschiedene Frameworks doch immer unterschiedlich mit Wachstum um. Interdependenzen im Code führen zu vermischten Strukturen und immer mehr Klassen die zwischen verschiedenen Teilanwendungen geteilt werden. Und oftmals wurde bei einer jungen Firma nun mal nicht die beste Technologie für die Zukunft gewählt, insbesondere weil die Zukunft schlichtweg unklar ist. Der Fokus liegt am Anfang schließlich im jetzt. Dann heißt es oftmals ``Refactor or Rewrite?'' \cite[vgl.][]{refactorrewrite}.

Die große Herausforderung besteht darin, entweder die bestehende Codebase wieder handhabbar zu machen, um Entwickler- und Betreiberproduktivität wieder zu erhöhen. Oder, idealerweise, ein System gar nicht erst in diesen Zustand kommen zu lassen. In beiden Fällen kann es hilfreich sein, auf den eigenen Architekturstil zu reflektieren.

Eine Architekturweise, die in den letzten Jahren immer mehr an Beliebtheit gewonnen hat, ist die ``Microservice Architektur''. Microservice Architektur verspricht, Code in kleine, gut zu handhabende Systeme aufzuteilen und somit klare Strukturen zu fördern.

In der vorliegenden Arbeit werde ich den Stil der Microservice Architektur vorstellen und die Vor- und Nachteile des Architekturstils benennen. Anhand eines praktischen Beispiels werde ich eine bestehende monolithische Anwendung um einen Microservice erweitern und meine Vorgehensweise erläutern.
Abschließend werde ich Schlüsse zur Umsetzung eines Anwendungsteils mit Hilfe der Microservice Architektur ziehen und das Pattern bewerten.
  \chapter{Microservice Architektur als Weg robuste und skalierbare Anwendungen zu entwickeln}
``Microservice Architektur'' beschreibt einen Stil der Softwareentwicklung, der vor Allem durch die Trennung einer großen Gesamtanwendung, in kleinere, separate Teile gekennzeichnet ist.~\footcite[vgl.][Seite 2]{newman2015building}
Entscheidend für die Microservice Architektur ist im wesentlichen die Abgrenzung zur monolithischen Anwendungsentwicklung.

\section{Microservices als Werkzeug zum Management von Komplexität}
Monolithische Anwendungen bestehen im wesentlichen aus einer einzelnen Einheit.~\footcite[vgl.][]{Fowler:Intro} In einer klassischen drei-Schichten Anwendung (Frontend, Backend, Persistence Layer)~\footcite[vgl.][]{MSDN:TTA} bietet es sich an, die gesamte Logik in einer Anwendung zu verwalten. Dies ist auch der Standardaufbau der meisten Webframeworks. Es gibt eine Schnittstelle zwischen der clientseitigen und der serverseitigen Anwendung, ebenso wie zwischen der serverseitigen Anwendung und der Datenbank. Unabhängig wie unübersichtlich ein System mit zunehmender Größe wird, das zusammenhalten dieser Anwendungsteile ist der default Weg. Schwer zu handhabende, komplexe und große monolithische Anwendungen entstehen häufig dann, wenn bei Wachstum mit der Zeit dieser Aufbau nicht überdacht wird.~\footcite[vgl.][]{infaktuell}

Anstatt einer großen Anwendung, deren einzelne Teile gemeinsam deployt werden, in einem gemeinsamen repository liegen und auf der Ebene der genutzten Programmiersprache kommunizieren, wird bei der Microservice Architektur die Gesamtanwendung hingegen in einzelne Teilanwendungen aufgesplittet. Diese liegen in separaten Repositories, können getrennt voneinander deployt werden und kommunizieren über externe Schnittstellen. Diese separaten Teilanwendungen sind meist anhand von Aufgabengebieten getrennt und identifiziert. Die Trennung nach Single Responsibility Principle~\footcite[vgl.][Seite 108]{Martin:SRP} wird hier von der Ebene des Codes in die Ebene der Gesamtarchitektur gehoben.

Die Aufteilung einer monolithischen Gesamtanwendung in kleinere Services ist jedoch keineswegs ein Allheilmittel für alle Probleme. Microservice Architektur birgt seine ganz eigenen Herausforderungen. Microservice Architektur bringt demnach vor allem Unterschiede mit sich. Die Frage ob sich diese positiv oder negativ auswirken, hängt ganz individuell von der bestehenden Anwendung, dem Entwicklerteam und den vorhandenen Resourcen ab.

Ein Architekturstil, der der Microservice Architektur ähnelt, ist die ``Service-Oriented Architecture''. Aufgrund der ähnlichen Ansätze, die dieser Architekturstil verfolgt, sollte er näher betrachtet werden.
Service-Oriented Architecture (SOA) hat sich ebenfalls entwickelt, um die Probleme großer, monolithischer Anwendungen zu lösen.~\footcite[][Seite 8]{newman2015building} Mit Hilfe von Services, soll wiederverwendbarer Code geschaffen werden, der dann von verschiedenen Endanwendungen genutzt werden kann. Die Kommunikation findet über Netwerke statt und nicht mehr über direkte Aufrufe im Code. Auch SOA soll es einfacher machen Code zu organisieren, zu strukturieren und bei Bedarf zu ersetzen. Warum entsteht nun also der Trend der Microservices, wenn SOA einen ähnlichen Ansatz verfolgt? Ein großes Problem von SOA liegt eigentlich in den damit verbundenen Technologien wie SOAP, vendor middleware und der nicht ausreichend klaren Trennung der Services.~\footcite[][Seite 8]{newman2015building} SOA Services teilen sich z.B. oft Datenbanken, wohingegen dies ein no-go bei Microservices ist. SOA scheint zudem nicht klar genug definiert zu sein.~\footcite[vgl.][]{Fowler:Intro} Daher ist es ratsam einen klarer definierten Begriff, wie den der Microserice Architektur anzustreben.~\footcite[][]{Fowler:Intro}
(FIX THIS IS NOT GOOD AT ALL)

\section{Herausforderungen und Potentiale von Microservices}
Wie jeder Architekturstil bieten auch Microservices sowohl Vorzüge, als auch ganz eigene Kosten.~\footcite[vgl.][]{Fowler:Guide} 
Zum Einen der Punkt separater, klar strukturierter Codebases. Diese verschaffen einen besseren Überblick, sodass sich Entwickler leichter in ein Aufgabe einarbeiten können ohne sich durch irrelevante Codeteile arbeiten zu müssen. Gerade in größeren Teams ist dies von großem Vorteil. Je mehr parallel an voneinander abhängigen Codeteilen gearbeitet wird, desto mehr merge Konflikte entstehen. Auch wenn die eigentlichen Aufgaben komplett voneinander getrennt sind, kommt es häufig zu Konflikten. Je vermischter der Code ist, desto mehr Konflitke enstehen. Ähnlich wie das Single Responsibility Principle auf Methoden- oder Klassenebene gilt, so kann man es auch auf Service Ebene als geltend ansehen. Getrennte, klar strukturierte Codeteile erhöhen die Wiederverwendbarkeit, erleichtern die Einarbeitung und das Management des Codes. Ebenso können Änderungen leichter eingearbeitet werden, da die Chance von Seiteneffekten reduziert ist. Komplexitäten werden demnach nicht per sé reduziert, sie werden jedoch aufgetrennt und verschoben. Somit werden sie leichter zu handhaben.
So lassen sich zum Einen kleinere, autonome Anwendungsteile aus Entwicklersicht besser verwalten. Die einzelnen Anwendungsteile müssen jedoch über eine externe Schnittstelle kommunizieren. Diese Aufrufe, die in der Regel über das Intra- oder Internet stattfinden sind aufwändiger als simple Code Calls. Wie diese API calls stattfinden ist nicht zwangsläufig vorgeschrieben [WAS FUR MOGLICHKEITEN GIBT ES]. Im Allgemeinen gilt jedoch: Verteilte Systeme sind schwieriger zu programmieren, da remote calls langsam und fehleranfälliger sind.~\footcite[vgl.][]{Fowler:Guide}

Die Aufteilung in separate Services kann ebenso das Deployment vereinfachen. Relativ kleine Änderungen können leichter deployt werden, da nicht das ganze System erneut deployt werden muss. Bei großen, komplexen Anwendungen kann dies ein ganz eigenes Risiko bilden. Ist das Deployment einer Anwendung riskoreich, wird im Allgemeinen seltener deployt. Features werden erst später deployt, doch mit wachsenden Unterschieden zwischen Deployments, wächst auch das Risiko von Fehlern.~\footcite[vgl.][Seite 6]{newman2015building}
Separate Deploys von Microservices machen die Versionsverwaltung von diesen jedoch um so wichtiger~\footcite[vgl.][Seite 62]{newman2015building}~\footcite[vgl.][]{Vergleichsartikel}. Breaking Changes können nicht einfach deployt werden und da gleichzeitige Deploys verschiedener, mit einander kommunizierender Services nicht möglich sind, müssen diese Deploys schrittweise und kontrolliert durchgeführt werden: Zunächst muss der Service um eine neue Version erweitert werden, die alte Schnittstelle darf hierbei nicht verändert werden. Meist werden hierzu API Versionen genutzt. So kann die erste Schnittstelle über die URL

\begin{lstlisting}[language=Ruby]
https://meinservice.tld/api/v1/eineRoute
\end{lstlisting}

angesprochen werden. Diese Schnittstelle wird von den Änderungen nicht tangiert. Die neue Schnittstelle umfasst sowohl die neuen, als auch alle alten, weiterhin gewünschten, Funktionen. Sie ist über die URL

\begin{lstlisting}[language=Ruby]
https://meinservice.tld/api/v2/eineRoute
\end{lstlisting}

anzusprechen. Nach Deploy dieser neuen Service Version, wird als nächstes der konsumierende Code geupdated. Hierbei wird die Nutzung der Funktionen auf die neue Version aktualisiert und die Request auf die neue URL umgeleitet. Alle Referenzen zur alten Route des konsumierten Services sollten hierbei entfernt werden. Abschließend kann, solange sichergestellt ist, dass die alte Service Version nicht mehr genutzt wird, der Service ein weiteres mal geupdated werden. Hierbei wird der Code der alten Version entfernt. Dies ist in der Regel nur bei firmeninternen APIs üblich. Externe APIs sollten in veralteten Versionen noch eine gewisse Zeit angeboten werden. Sollte in der neuen API Version ein Fehler auftreten, muss, vorausgesetzt v1 ist unverändert zu erreichen, nur der konsumierende Service gerollbacked werden. Datenbankmigrationen müssen hier nach dem gleichen Prinzip ebenso in mehreren Schritten durchgeführt werden. Das deployment kann also in vielen Fällen bei Microservices leichter sein, gestaltet sich bei großen Änderungen jedoch schwieriger.

In diesem Zug sollte auch das Thema Testing angesprochen werden. Testen, gerade über die Grenzen von Services hinweg, ist durchaus eine Herausforderung die es zu meistern gilt (FIX FORMULIERUNG). Auf der anderen Seite tendieren große monolitische Anwendungen dazu, eine lange Testlaufzeit zu haben. Ab einer gewissen Größe ist es kaum noch möglich, dass der Entwickler alle Tests lokal ausführt. Der Einsatz von Continuous Integration schafft hier natürlich Abhilfe, doch auch hier sind die Ressourcen nicht unbegrenzt. Bei einer Vielzahl von Entwicklern, die parallel arbeiten, können auch mit mehreren concurrent builds Buildstaus oft nicht vermieden werden. Separate Codebases mit separaten kleineren Tests, haben hier vor Allem den großen Vorteil, dass nicht alle Tests immer erneut ausgeführt werden müssen.

Somit verhält sich die Gesamttestlaufzeit t stark unterschiedlich. Die Parallelität des Continuous Integration Servers vernachlässigend, berechnet sich die Gesamtlaufzeit der Tests wie folgt:

$ t = \displaystyle\sum_{i=1}^{noOfCommits} Tt(codebase_i) $

Die Gesamttestlaufzeit berechnet sich demnach additiv aus den Einzeltestlaufzeiten (Tt) der jeweiligen Codebases. Da die Einzeltestlaufzeit bei Microserices wesentlich geringer ist, kann hier enorm an Zeit gespart werden.

Bei vier Entwicklern und einer monolithischen Testlaufzeit von zehn Minuten erreicht man eine Gesamttestdauer von 40 Minuten

$ t = \displaystyle\sum_{i=1}^{4} Tt(codebase_i) = 10m + 10m + 10m + 10m = 40m $

Bei identischer Codebase, beliebig verteilt auf vier Microservices mit identischer Gesamttestdauer von zehn Minuten, erreicht man jedoch nur eine Testdauer von zehn Minuten 

$ t = \displaystyle\sum_{i=1}^{4} Tt(codebase_i) = 3m + 4m + 1m + 2m = 10m $

Natürlich muss man zu der reinen Testdauer der Codebase einen gewissen Overhead im CI System einplanen, der nicht proportional zur Testdauer ist, dieser ist aber für die Ergebnisse zu vernachlässigen. Ebenso sollte die Zahl von Testfällen bei Microservice Architektur etwas erhöht sein, aber auch dies sollte im Vergleich zu einer großen monolithischen Anwendung noch zu einer großen Ersparnis führen.

Separate Services ermöglichen auch, komplett verschiedene (FIX DIVERSE?) Technologiestacks einzusetzen. Im Gegensatz zur Erweiterung einer monolithischen Anwendung, muss hier theoretisch nur bedingt auf die bereits eingesetzten Technologien geachtet werden. Ein separater Service mit separater Datenbank kann durchaus eine andere Datenbanktechnologie verwenden. Da über den separaten Service ohnehin ein separates Datenbankadapter implementiert werden muss, ist es aus reiner Implementierungssicht nicht nachteilig eine andere Datenbanktechnologie zu wählen. Hierbei kann durchaus auf die für den Anwendungsfall spezifischen Optimierungsmöglichkeiten geachtet werden. Ebenso kann eine für den Anwendungsfall optimierte Programmiersprache gewählt werden.
Das diese diverse (FIX ENGLISCH DIVERSE) Technologiewahl aus technischer Sicht möglich und ratsam ist, heißt jedoch keineswegs, dass sie tatsächlich so gewählt werden sollte. Die bestehenden Technologien in einem Unternehmen, welches beschließt eine große, bestehende Anwendung in Microservices aufzuteilen oder um einen Microservice zu erweitern, sind in der Regel tried and tested. Eine Vielzahl der Entwickler des Unternehmens wird mit den bestehenden Technologien vertraut sein. Fällt ein Entwickler aus, sind vermutlich noch genug andere Entwickler mit ähnlichen Fähigkeiten vorhanden um in dringenden Fällen die Arbeit zu übernehmen. Wählt man nun aber eine neue Programmiersprache, eine neue Datenbanktechnologie und neue Monitoring Tools aus, so ist dies ggf. nicht nur mit erhöhter Einarbeitungszeit verbunden, sondern auch mit höheren Managementkosten. Müssen neu eingstellte Entwickler nun den gesamten Technologiestack beherrschen oder nur einen der zwei Teile? Sind immer ausreichend Entwickler vorhanden um einen ausfallenden Entwickler zu kompensieren? Was passiert wenn der Entwickler des neuen Go Microservices kündigt und kein anderer Entwickler Go beherrscht? Die technologische Heterogenität kommt demnach zu einem Preis. Man sollte genau abwägen, bis zu welchem Grad die Diversifizierung des Technologiestacks lohnenswert ist.~\footcite[vgl.][Seiten 5, 6]{newman2015building}
Die Definition eines klaren, ``erforderten Standards''~\footcite[vgl.][Seiten 20, 21]{newman2015building} bietet sich hierzu an. Twitter und Netflix limitieren hierzu z.B. auf Technologien die unter der Java Virtual Machine (JVM) laufen (FIX FORMULIERUNG).~\footcite[][Seite 6]{newman2015building} Entwickler können so eine neue Programmiersprache wie Scala oder JRuby wählen, es wird jedoch auf bekannte Technologien im Betrieb des Servers gesetzt. Dieser erforderte Standard bezieht sich auch auf die gewählten Wekzeuge zum Monitoring und die eingesetzten Schnittstellentechnologien.~\footcite[vgl.][Seite 21]{newman2015building}

Weiterhin ermöglichen separate Services auch das separate Skalieren der Anwendung. Bei Monolithen beschränkt sich die Skalierung im Allgemeinen auf Load Balancing~\footcite[vgl.][]{infaktuell}. Hierzu wird eine Anwendung auf mehrere Server dupliziert, die Anwendungsreplikationen sind hierbei funktional identisch. Ein Load Balancer verteilt dann, nach bestimmten Regeln, die ankommenden Requests auf verschiedene Server.~\footcite[vgl.][]{loadbalancing} Diese verteilte Last ist jedoch unabhängig von den einzelnen Anwendungsteilen. Diese können eine unterschiedliche Individuallast haben, dies kann jedoch über den Load Balancer nicht beachtet werden.
Bei Microservices hingegen kann mintuiöser skaliert werden (fn ANHANG QUELLE)... (FIX AUSBAUEN)

Einzelne Services bedeuten aber nicht nur die Möglichkeit von granulärer Kontrolle, sondern auch die Pflicht der separaten Überwachung. Wo monolithische Anwendungen eine Quelle von Metriken, eine Anwendung zu überwachen und eine Anwendung zu deployen haben, haben Microservices viele.~\footcite[vgl.][]{Heroku:GoMicro}

Weiterhin kommt die Verteilung von Services mit ihren ganz eigenen Kosten. Verteilte Systeme haben den direkten Nachteil, das sie verteilt sind.~\footcite[][]{microtradeoffs} Zwar haben wir klar getrennte Systeme, die von der Codebase her gut zu managen sind, remote calls sind aber immer fehlerbehaftet. Der Service kann down sein, die Antwort zu lange dauern und zu einem Timeout führen und dauern im Allgemeinen länger als code calls. Timeouts und langsame Serviceantworten können natürlich über asynchrone Aufrufe gelöst werden, aber asynchrone Aufrufe bringen ihre eigenen Probleme mit sich. Nicht umsonst spricht man hier oft von der Callback-Hell.~\footcite[vgl.][]{callbackhell} Zweifelsohne kann man Die Callback-Hell vermeiden, die Integration einer Netzwerkschnittstelle in die Codebase birgt aber immer Gefahren. Man kann schlichtweg nicht von einem sicheren Netztwerk ausgehen, demnach bedarf es einem Sicherheitsmechanismus zur Authentifizierung. Bandbreite kann ggf. mit Kosten verbunden sein und ist nicht unbegrenzt verfügbar, das Netztwerk kann fehlerbehaftet sein (FIX RELIABILITY).~\footcite[vgl.][]{distributedfallacies}

Zu guter Letzt sollte man die reine Entwicklungszeit betrachten. Zwar ist die separate Codebase im Nachhinein ggf. leichter zu verwalten, die initiale Entwicklung ist aber durchaus mit einem gewissen Mehraufwand verbunden. Die Wahl der optimalen Technologien ist mit Recherche verbunden, der Einsatz neue Technologien darüber hinaus mit einer Einarbeitungszeit. Das initiale Setup von Programmiersprache, Framework und Datenbank ist ein zusätzlicher Aufwand, der bei der Erweiterung einer bestehenden Codebase nicht anfällt. Hinzu kommt außerdem die Einrichtung des Deployments. Git, Continuous Integration / Deployment, Servereinrichtung, sowie das Monitoren und Warten der Server sind vor allem zum Beginn des Lebenszyklus eines neuen Microservices ein nicht unerheblicher Mehraufwand. Dies und andere Herausforderungen gegen die Potentiale der Microservice Architektur abzuwägen, muss im konkreten Einzelfall immer individuell geschehen.

\section{Der Weg zum Microservice}
Wie eine Anwendung, bestehend aus verschiedenen Microservices, entsteht und entwickelt wird, hängt von der konkreten Ausgangssituation ab. Hier kann man in verschiedenen Situationen unterscheiden:
\begin{description}
  \item[Aufsplittung] \hfill \\
  Ein bereits bestehender Anwendungsteil soll zu einem Microservice umgebaut werden. Der bestehende Code soll nicht ersetzt, sondern lediglich verschoben werden. Die Schnittstellen zum bestehenden Code müssen erneuert werden.
  \item[Rewrite] \hfill \\
  Ein bereits bestehender Anwendungsteil soll von Grund auf neu geschrieben werden. Bestehender Code wird ersetzt und der neue Code als separeter Service in die bestehende Anwendung integriert.
  \item[Erweiterung] \hfill \\
  Ein neuer Anwendungsteil soll entwickelt werden. Die Aufgaben des neuen Anwendungsteils sind weder in der bestehenden Anwendung, noch extern vorhanden. Es wird neuer Code geschrieben und in die bestehende Anwendung integriert.
\end{description}
Zwar unterscheiden sich die drei aufgeführten Ausgangssituationen im Endprodukt nicht zwangsläufig. In der Entwicklung des Microservices muss aber grundlegend anders vorgegangen werden. 

Der erste und zweite Fall unterscheiden sich nicht grundlegend. Die bestehende Schnittstelle zur betroffenen Funktion muss komplett überarbeitet werden. Sowohl in der bestehenden, als auch in der neu entstehenden Anwendung muss diese Schnittstelle geschaffen werden. Auch wenn nur existierender Code verschoben wird, muss eine komplett neue Schnittstelle geschaffen werden. Wo vorher code calls genutzt wurden, werden nun API calls genutzt.
Welches dieser zwei Vorgehensweisen mit mehr Aufwand verbunden ist, hängt zum Großteil von der Komplexität und Struktur der bestehenden Anwendung ab. Auch wie modular der bereits existierende Code ist, ist entscheidend. Je nach Alter des bestehenden Codes ist eine Neuentwicklung teils ratsam. Da die Umarbeitung ohnehin mit erheblichem Arbeitsaufwand verbunden ist, bietet sich eine Optimierung des bestehenden Codes häufig an. Bei großen, komplexen Legacy Anwendungen ist dies in der Praxis aber häufig nicht ratsam.
Bei der Aufsplittung des Monolithen bietet es sich hierbei an, sich am Saum (seams ~\footcite[vgl.][Seite 29 ff.]{feathers2004working}) orientiert werden. Codeteile, die isoliert sind und ohne Seiteneffekte geändert werden können, eignen sich besonders gut um in separate Services verschoben zu werden. Natürlich ist es alles eine Frage der konkreten Problematik, ob ein Service frei, neu gewählt werden kann oder aus Notwendigkeit, unabhängig vom Grad der Komplexität, geschaffen werden muss.

Im dritten Fall kann der neue Anwendungsteil nach idealen und neuen Vorstellungen entwickelt werden. Zwar muss auch hier eine Integration zum bestehenden Code geschaffen werden, es gibt aber keine bisherige Implementation die beachtet werden muss. Die Entwickler haben die freie Wahl, die bestmögliche Integrationsform zu wählen. Dies kann häufig zum besten Endprodukt führen, da keine Einschränkungen durch bestehenden Legacy Code die Programmierung beeinträchtigen. Natürlich wird hier aber ein komplett neues Feature entwickelt. Die Konzeption der Architektur, des Codes und die komplett neue Entwicklung können hier mit einem wesentlichen Mehraufwand verbunden sein. Die Integration in die bestehende Anwendung sollte aber in der Regel mit weniger Problemen verbunden sein.
  \chapter{Entwicklung einer real-time Query Anwendung zur Beschleunigung von Userprofilqueries}
Ein wesentlicher Bestandteil der bestehenden Betriebsanwendung bildet das Abfragen von Userprofildaten. Um für Umfragen passgenau Teilnehmer auszuwählen, gibt es in der Profildatenbank, die knapp 400.000 Nutzer umfasst, 176 Profilfelder. Diese können in allen denkbaren Kombinationen abgefragt werden. Die besteheden Datenbankabfragen geschehen aufgrund ihrer langen Laufzeit asynchron. Hier soll eine neue, synchrone Schnittstelle geschaffen werden.

\section{Real-time Anforderungen erzwingen eine neue Entwicklung}
Das das Finden von Umfrageteilnehmern und demnach das Abfragen der Profilfelder einen wesentlichen Kern der Betriebsanwendung bildet, besteht dieser Teil der Anwendung mit am längsten. In den fast 3 1/2 Jahren ist sowohl die Anzahl der User, also auch die Anzahl der Profilfelder enorm gewachsen, weitaus mehr als initial erwartet. Im Laufe der Zeit stellte sich heraus, das die gewählte Datenbanktechnologie MongoDB, das gewählte ``Schema'' (FIX NO SCHEMA!) und die Art wie gequeried wird, nicht optimal und daher nicht schnell genug sind. Da das Abfragen der Profilfelder wesentlichster Bestandteil des Kerngeschäfts ist und die Defizite die Situation mit wachsender User- und Kundenzahl nur verschärfen, wurde deutlich, dass eine Überarbeitung der bestehenden Strukturen notwendig ist.

\section{Majestic Monolith vs Minimal Microservice}
Um die Probleme zu lösen kann auf vielerlei Wege vorgegangen werden. Eine neue Datenbanktechnologie mit optimiertem Schema kann gewählt werden, der Code kann auf Geschwindigkeit optimiert oder komplett neu geschrieben werden.
Die hier eingesetzte Datenbank MongoDB wird noch für weitere Funktionen in der Anwendung genutzt und kann daher nicht komplett ersetzt werden. Eine bestehende MySQL Datenbank ist bereits als Datawarehouse im Einsatz und erfährt bereits load.
Die bestehende Anwendung ist mit den Jahren allgemein recht groß und unhandlich geworden.
Eine Aufsplittung der Anwenung in mehrere Teile, mit Hilfe der Microservice Architektur, wurde beschlossen.
Mit Hilfe der Microservice Architektur können hier Probleme der Skalierung, Optimierung der Datenbanktechnologie, Beschleunigung des Codes und Schaffung von Ordnung im Code angegangen werden.
Da eine neue Datenbanktechnologie und ein komplett separater Service ohnehin mit vielen Änderungen verbunden ist, wurde beschlossen den alten Code nicht in einen neuen Service zu migrieren, sondern die Funktionalität neu zu Schaffen.

\section{Mit dem Strangler Pattern vom Monolithen zur Microservice Architektur}
Da die aktuelle Arbeitslage, die vorhandenen Entwickler und die Größe der Anwendung es nicht zulassen diese direkt komplett zu überarbeiten, wird hier die Architektur mit Hilfe des Strangler Patterns~\footcite[][]{Fowler:Strangler} Schritt für Schritt umgesetzt.
Statt traditionellem Load Balancer, wird hier mit Hilfe des Ruby Flipper Gems~\footnote{https://github.com/jnunemaker/flipper} nach und nach mehr Last auf den neu entstehenden Service verteilt.~\footcite[vgl.][]{Hammant:Strangler}
  \chapter{Implementierung aufbauend auf bestehenden Technologien, REST API und Continuous Delivery}

\section{Ruby / Sinatra als schlanke Ergänzung des bestehenden Technologiestacks}
Wie bereits in vorangegangenen Kapiteln beschrieben, bieten Microservices die Möglichkeit zum optimierten Einsatz von Technologien. Für die zu entwickelnde Anwendung gab es diverse Optimierungsmöglichkeiten. Die Hauptanwendung ist im Ruby Framework Ruby on Rails\footnote{http://rubyonrails.org} entwickelt worden. Ruby on Rails ist jedoch als Framework zu heavy-weight und mit zu viel Overhead verbunden, als das es sich für einen schnellen, minimalistischen Microservice eignen würde. Ruby on Rails ist an erster Stelle für monolithische Anwendungen entwickelt.~\footcite[][]{rails:doctrine}
Hierbei ist nicht nur die Performance entscheident, sondern auch die Struktur des Codes. Rails als traditionelles Model-View-Controller Framework\footcite[][]{wiki:mvc} eignet sich somit vor Allem auch nicht aufgrund seiner Struktur. Die Rails API Variante\footnote{https://github.com/rails/rails/pull/19832} hingegen hat immer noch zu viel overead für eine optimierte Schnittstelle.

Alternativen bilden sogenannte Microframeworks\footcite[][]{wiki:micro}. Microframeworks zeichnen sich im Gegensatz zu full-stack Frameworks dadurch aus, das viele der Funktionen nicht Teil des mitgelieferten Umfangs sind. In den meisten Sprachen gibt es diverse Microframeworks, wie z.B. Flask\footnote{http://flask.pocoo.org} für Python, Express\footnote{http://expressjs.com} für Node, Sparkjava\footnote{http://sparkjava.com}, oder das Sinatra Framework\footnote{http://www.sinatrarb.com} für Ruby. Die Sprache Go\footnote{https://golang.org} kommt bereits mit gut ausgebauten net/htttp Paketen und umfasst dadurch die meisten üblichen Funktionen schon ohne Framework.

Zwar gibt es Geschwindigkeitsunterschiede in diesen Frameworks\footcite[vgl.][]{frameworks}, im Vergleich zu klassischen full-stack Frameworks sind diese aber unerheblich. Da die Datenbank in der bestehenden Anwendung den größten Flaschenhals bildet (FIX STATS), muss hier nicht zwangsläufig das beste Framework gewählt werden. Stattdessen sollte auf die bestehende Firmenstruktur geachtet werden.

\section{Schaffung einer standardisierten REST Schnittstelle}

\section{Optimierung des Lesezugriffs auf die bestehenden Daten}

\section{Mit Testing, Continuous Integration und Cloud Hosting zu Continuous Delivery}

\section{Betrieb der Anwendung auf AWS - Betreuung, Monitoring und Scaling}


(FIX IDEEN FÜR KAPITELUMSTRUKTURIERUNG)

Sinatra und PostgreSQL zur Optimiertung des bestehenden Technologiestacks
Schaffung einer standardisierten REST Schnittstelle
Integration des Microservices mit dem Monolithen
Betrieb der Anwendung auf AWS - Betreuung, Monitoring, Scaling
  \chapter{Microservice Architektur als Pattern der Zukunft?}

Die Frage, ob Microservice Architektur im Betrieb eingesetzt werden sollte, lässt sich pauschal schlichtweg nicht beantworten. Die Voraussetzungen um erfolgreich zu einer Microservice Architektur zu wechseln, sind individuell und vor Allem von der bestehenden Anwendung und den Teamgegebenheiten abhängig. Microservices können dabei Helfen viele Probleme des klassischen monolithischen Anwendungsentwicklung, insbesondere eine große, unübersichtliche Codebase, Performance Probleme, hohe Kosten und unklare Strukturen, zu lösen. Jedoch setzt der Einsatz vieles Voraus, dessen Aufbau ein langwieriger Prozess sein kann. Eine Teamstruktur, in denen sich Entwickler selbst für den Betrieb der von ihnen entwickelten Anwendung verantwortlich sehen, dynamische Serverstrukturen  zum Skalieren der Services wie sie z.B. AWS, Google Compute Engine und Heroku bieten, automatisierte Deploys und ausgereifte Monitoring Lösungen sind für den erfolgreichen und sinnvollen Einsatz von Microservices unabdingbar.

Sollten Microservices nicht direkt zu Beginn des Softwarelebenszyklus entstehen, ist es wichtig den Übergang so kontrolliert wie möglich zu schaffen. Die Microservice Architektur stellt sicherlich keine schnelle Lösung für Probleme im Betrieb dar. Stattdessen ist sie eine längerfristige Strategie um die Organisation und Arbeit in der Firma zu verbessern.\cite{newman2015building} Wird eine bestehende Anwendung zu einer Microservice Architektur umgewandelt, so bietet es sich zum Beispiel an, neu entstehende Software Teile als Microservices zu entwickeln, um so die Codebase nicht noch weiter wachsen zu lassen. Auch besteht die Möglichkeit einen neuen Service zunächst an den bestehenden Hauptservice zu koppeln. Das ``große'' Framework Ruby on Rails bietet so z.B. die Möglichkeit, mit dem Microframework Sinatra entwickelte Apps in der Hauptanwendung zu \textit{mounten}. Hierdurch erhält die Subanwendung einen relativen Pfad in der Hauptanwendung und kann so wie über interne Routen angesprochen werden. Die Subanwendung kann dann mit der Hauptanwendung deployt und betrieben werden. Hierdurch können viele der initialen Herausforderungen von Microservices, wie das Monitoring, der separate Betrieb und die dynamische Skalierung der Anwendung, umgangen werden, jedoch einige der Vorzüge, wie separate Codebases und klare Strukturen bereits genutzt werden. Ebenso können hier dann z.B. auch queueing-Dienste wie Sidekiq als Kommunikationsschnittstelle dienen.~\cite[vgl.][]{sidekiqmessaging} Dies erspart zunächst die Entwicklung einer vollwertigen API. Nach Aufbau geeigneter Firmenstrukturen kann die Anwendung dann mit verhältnismäßig geringem Aufwand zu einem vollwertigen Microservice ausgebaut werden. So ist man zeitlich bei der Programmierung eines Microservices nicht an dessen unmittelbaren Betrieb als solcher gebunden.

In der hier entwickelten Anwendung wurde vor allem beim Ersetzen der bestehenden Anwendungsteile durch den Microservice auf eine kontrollierte Integration geachtet. Durch Feature Toggles wurden zunächst im Betrieb nur bestimmte, explizit gewählte Queries an den Microservice gesendet. Später können dann schrittweise einzelne User freigeschaltet und prozentual, langsam eine immer höhere Zahl an Queries an den Microservice umgelagert werden. So kann nach Testen und Sicherstellen der Funktion langsam immer mehr Last auf den Microservice verschoben und so die Chance einer Überlastung und anderer unerwarteter Fehler minimiert werden.

In der hier umgesetzten praktischen Arbeit kann man den entwickelten Microservice als Erfolg betrachten. Der entwickelte Microservice ist weit mehr als nur ein Prototyp und wird nun aktiv in den Betrieb integriert. Zwar ist die Entwicklung hier noch nicht abgeschlossen und die parallele Anwendung mit der alten Struktur ist durchaus ratsam, die erhofften Performance Verbesserungen durch die Entwicklung der neuen Teilanwendung konnten jedoch erreicht werden. Die Transition, wenn auch noch nicht komplett abgeschlossen, läuft erfolgreich und die Struktur der Anwendung konnte, vor allem in Augen der betroffenen Entwickler, verbessert werden. Lediglich auf der Seite der Kosten konnte keine Verbesserung erzielt werden. Dies liegt vor allem daran, das der bisherige Hosting Plan keinerlei Dynamik vorsieht und so, trotz sinkender Last, zu keiner Kostenreduzierung führt. Die durch den neuen Service entstehenden Mehrkosten sind aber nicht höher als erwartet und somit stellt auch dies kein Problem dar.

Die Ausarbeitung von neuen Monitoring, Logging und Deploy Lösungen wurde im Rahmen der Arbeit erfolgreich umgesetzt und bereits im Unternehmen erprobt.

Aus Entwicklersicht kann man ebenfalls von einem Erfolg sprechen. Die Entwicklung eines komplett neuen Anwendungsteils ist  angenehmer als eine bestehende, drei Jahre alte Anwendung um einen weiteren Teil zu ergänzen. Es bestehen keine Einschränkungen an die der Entwickler gebunden ist und vor Allem durch die strenge Trennung in einen separaten Service und die Definition von eindeutigen Schnittstellen kann die interne Definition vom Entwickler frei gewählt werden. Die klare Trennung und das Ansprechen über eine REST API machen den Code sauber und gut nachzuvollziehen. In der Zukunft wird es hier möglich sein den neuen Codeteil leicht zu warten, ohne unerwartete Nebeneffekte zu erfahren. Die Einbindung des neuen Services in die bestehende Anwendung stellte sich jedoch als weitaus schwieriger als erwartet heraus. Hier fanden sich genau all die Probleme, die Microservices zu vermeiden hoffen. Durchmischte Schnittstellen mit Seiteneffekten bei Änderungen führten dazu das zeitweise mehrere Hundert Testfehler auftraten. Die Integration, nach vermeintlicher Fertigstellung des Services, dauerte einige Wochen.
Aufgrund der erhöhten Integrationszeit konnte bisher auch kein automatisches Updaten der Daten im Microservice eingerichtet werden. Die Schnittstelle zum Einführen von Userdaten ist jedoch im Microservice fertiggestellt und wurde bereits im Betrieb erprobt. Bisher wurden jedoch nur manuell Daten über die Schnittstelle in \textit{gebatchter} Form in den Microservice übertragen.

Während der Entwicklung des Microservices wurde auch schnell deutlich, dass es sich um einen relativ jungen Architekturstil handelt. Vieles was bei internen Anwendungsteilen keine Rolle spielt, muss hier noch selbst implementiert werden. Zwar bieten Toolkits wie Go kit\cite{gokit} Schnittstellendefinitionen, Circuit Breaker und Rate Limiter, sowie Metrics, Logging und Request Tracing aber auch dieses Projekt ist noch sehr jung und bei weitem keine standardisierte Lösung. Andere Programmiersprachen besitzen zudem teilweise noch keine solchen Toolsets in diesem Bereich.

Die vorliegende Arbeit zeigt, das die Integration in ein bestehendes System möglich ist und auf lange Sicht Vorteile hat, jedoch ist sie auf Kurze Sicht mit nicht unerheblichem Mehraufwand verbunden. Ob die in der Entwicklung entstandenen Mehrkosten jedoch über Zeit eingespart werden können, relativ über leichtere Entwicklungsarbeit dank sauberer Codebase oder durch bessere Performance des Systems, wird sich mit der Zeit zeigen und muss immer im konkreten Einzelfall evaluiert werden.

  %\begin{appendix}


\chapter{Code Listings}

Code zum Testen der Performance MongoDB vs Microservice

\begin{lstlisting}[language=Ruby]
mongoid_results_count  = []
mongoid_results_select = []
prophet_results_count  = []
prophet_results_select = []

seventy_five_zips = ["24943", "24944", "24945", "24946", "24947",
    "24948", "24949", "24950", "24951", "24952", "24953", "24954",
    "24955", "24956", "24957", "24958", "24959", "24960", "24961",
    "24962", "24963", "24964", "24965", "24966", "24967", "24968",
    "24969", "24970", "24971", "24972", "24973", "24974", "24975",
    "24976", "24977", "24978", "24979", "24980", "24981", "24982",
    "24983", "24984", "24985", "24986", "24987", "24988", "24989",
    "24990", "24991", "24992", "24993", "24994", "24995", "24996",
    "24997", "24998", "24999", "25000", "25001", "25002", "25003",
    "25004", "25005", "25006", "25007", "25008", "25009", "25010",
    "25011", "25012", "25013", "25014", "25015", "25016", "25017"]
    
three_hundred_zips = ["24943", "24944", "24945", "24946", "24947",
    "24948", "24949", "24950", "24951", "24952", "24953", "24954",
    "24955", "24956", "24957", "24958", "24959", "24960", "24961",
    "24962", "24963", "24964", "24965", "24966", "24967", "24968",
    "24969", "24970", "24971", "24972", "24973", "24974", "24975",
    "24976", "24977", "24978", "24979", "24980", "24981", "24982",
    "24983", "24984", "24985", "24986", "24987", "24988", "24989",
    "24990", "24991", "24992", "24993", "24994", "24995", "24996",
    "24997", "24998", "24999", "25000", "25001", "25002", "25003",
    "25004", "25005", "25006", "25007", "25008", "25009", "25010",
    "25011", "25012", "25013", "25014", "25015", "25016", "25017",
    "25018", "25019", "25020", "25021", "25022", "25023", "25024",
    "25025", "25026", "25027", "25028", "25029", "25030", "25031",
    "25032", "25033", "25034", "25035", "25036", "25037", "25038",
    "25039", "25040", "25041", "25042", "25043", "25044", "25045",
    "25046", "25047", "25048", "25049", "25050", "25051", "25052",
    "25053", "25054", "25055", "25056", "25057", "25058", "25059",
    "25060", "25061", "25062", "25063", "25064", "25065", "25066",
    "25067", "25068", "25069", "25070", "25071", "25072", "25073",
    "25074", "25075", "25076", "25077", "25078", "25079", "25080",
    "25081", "25082", "25083", "25084", "25085", "25086", "25087", 
    "25088", "25089", "25090", "25091", "25092", "25093", "25094",
    "25095", "25096", "25097", "25098", "25099", "25100", "25101",
    "25102", "25103", "25104", "25105", "25106", "25107", "25108",
    "25109", "25110", "25111", "25112", "25113", "25114", "25115",
    "25116", "25117", "25118", "25119", "25120", "25121", "25122",
    "25123", "25124", "25125", "25126", "25127", "25128", "25129",
    "25130", "25131", "25132", "25133", "25134", "25135", "25136",
    "25137", "25138", "25139", "25140", "25141", "25142", "25143",
    "25144", "25145", "25146", "25147", "25148", "25149", "25150",
    "25151", "25152", "25153", "25154", "25155", "25156", "25157",
    "25158", "25159", "25160", "25161", "25162", "25163", "25164",
    "25165", "25166", "25167", "25168", "25169", "25170", "25171",
    "25172", "25173", "25174", "25175", "25176", "25177", "25178",
    "25179", "25180", "25181", "25182", "25183", "25184", "25185",
    "25186", "25187", "25188", "25189", "25190", "25191", "25192",
    "25193", "25194", "25195", "25196", "25197", "25198", "25199",
    "25200", "25201", "25202", "25203", "25204", "25205", "25206",
    "25207", "25208", "25209", "25210", "25211", "25212", "25213",
    "25214", "25215", "25216", "25217", "25218", "25219", "25220",
    "25221", "25222", "25223", "25224", "25225", "25226", "25227",
    "25228", "25229", "25230", "25231", "25232", "25233", "25234",
    "25235", "25236", "25237", "25238", "25239", "25240", "25241",
    "25242"]

selectors = [
    { 
        :"profile.lifestyle.lifestyle_eu_birthplace.value"=>"yes",
        :"profile.education.highest_degree.value"=>"abitur" 
    },
    { 
        :"id" => "560149a622f009176300004f" 
    },
    { 
        :"profile.basic.zip_code.value".in => seventy_five_zips 
    },
    { 
        :"profile.basic.zip_code.value".in => three_hundred_zips
    },
    { 
        :"profile.basic.zip_code.value".in => seventy_five_zips,
        :"profile.lifestyle.lifestyle_eu_birthplace.value"=>"yes", 
        :"profile.education.highest_degree.value"=>"abitur" 
    },
    { 
        :"profile.basic.zip_code.value".in => three_hundred_zips,
        :"profile.lifestyle.lifestyle_eu_birthplace.value"=>"yes", 
        :"profile.education.highest_degree.value"=>"abitur" 
    },
    { 
        :"profile.basic.born_at.value" => Date.new(1989, 10, 30) 
    },
    { 
        :"id".ne => "560149a622f009176300004f" 
    },
    { 
        :"id".ne => nil
    }
]

selectors.each do |selector|
  puts "MongoidQuerier count"
  timing = Benchmark.measure do
    100.times do
      mq = MongoidQuerier.new(selector, nil)
      mq.population
    end
  end

  puts timing
  mongoid_results_count.push(timing)

  puts "MongoidQuerier pluck"
  timing = Benchmark.measure do
    100.times do
      mq = MongoidQuerier.new(selector, nil)
      mq.response_rates
    end
  end

  puts timing
  mongoid_results_select.push(timing)

  puts "ProphetQuerier count"
  timing = Benchmark.measure do
    100.times do
      pq = ProphetQuerier.new(selector)
      pq.population
    end
  end

  puts timing
  prophet_results_count.push(timing)

  puts "ProphetQuerier pluck"
  timing = Benchmark.measure do
    100.times do
      pq = ProphetQuerier.new(selector)
      pq.response_rates
    end
  end

  puts timing
  prophet_results_select.push(timing)
end

\end{lstlisting}

\end{appendix}



  \backmatter
  \markboth{}{}


  %\include{danksagung}
  
  \printbibliography

\end{document}
